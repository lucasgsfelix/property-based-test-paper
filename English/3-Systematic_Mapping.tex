\section{Systematic Mapping} \label{sec:sys_map}
	
	In this section, we present the developed Systematic Mapping. 

	\subsection{Protocol}

		The Systematic Mapping (SM) aim to provide a overview of a given field through a rigorous process of gathering and evaluation of works in the literature~\cite{petersen2008systematic}. Throught the results of a SM hopes to understands the tendencies of a researched field. To guide a work, some research question and objectives are raised. Next, we present our aim with this work, and the research questions that guide this paper.

		\begin{itemize}
			\item \textbf{Main Objective:} Identify the main studies of Property-Based Testing, and the applications until this moment.
			\item \textbf{Research Questions:}
			\begin{enumerate}
				\item What are the tools that suport this type of test ?
				\item Which branch of Property-Based Testing are most used/studied ?
				\item In which domains are most applyied ?
				\item What are Benefits and disadvantages of using Property-Based Testing ?
				\item What are the biggest challenges for the researchers in this field ?
			\end{enumerate}

		\end{itemize}

		Starting from the Main Objective and from the research questions was defined the strategies to search and selection of studies.

	\subsection{Studies Search}

		The strategy of searching was defined according to the bellow topics:
		\begin{itemize}
			\item \textbf{Source Selection Criteria:} The criteria for selection of sources used to retrieve papers follows the main conferences, \textbf{periodics} and symposiums where Software Test works are published. 
			\item \textbf{Research Methods:} Is used a automatic search where the Search-String is executed in the selected sources.
			\item \textbf{Keywords:} Test, Software Testing, Property-Based Testing
			\item \textbf{Search-String:} "Property-Based Testing" OR "Property Based Testing" OR "Property Based Test" OR "Property-Based Test" 
			\item \textbf{Selected Sources:} Scopus ~\cite{scopus}, ACM~\cite{acm}, IEEE~\cite{ieee}, Elsevier-Science Direct~\cite{esd}
		\end{itemize}

	\subsection{Studies Selection}
		
		To studies selection we define some criteria for work inclusion and exclusion which are defined bellow.

		\begin{itemize}
			\item Works on the field of Software Testing.
			\item Works on the field of Property-Based Testing.
			\item Works which are portuguese or English.
			\item Works with complete text avaliable on web.
			\item Works that has a abstract.
		\end{itemize} 


	\subsection{Studies Extraction}

		In this step was selected the main works in Property-Based Testing. In this proccess, was identified relevant works according to abstract, key-words and title. In this step duplicated works was excluded. The importance of each selected work was defined by reading the complete work verifying if the paper comply with the previously established requisites.

	\begin{table}[htbp]
		\centering
		\caption{Systematic Mapping Results}
		\label{tab:resultados}
		\begin{tabular}{|c|c|c|c|}
		\hline
		\textbf{Source} & \textbf{Returned} & \textbf{Selected} \\ \hline
			IEEE & 34 &  11  \\ \hline
			Scopus & 153 &  77  \\ \hline
			ACM & 55 &  11  \\ \hline
			Science-Direct & 29 & 2  \\ \hline
			\textbf{Total} & \textbf{271} & \textbf{101} \\ \hline
		\end{tabular}
	\end{table}

